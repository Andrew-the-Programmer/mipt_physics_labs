\documentclass{report}
\usepackage[fontsize=13pt]{scrextend}

\usepackage{my_lab}

\begin{document}

\graphicspath{{figures}}

\LabTitle{3.1.3}{Измерение магнитного поля земли}

\begin{document}
\textbf{Цель работы:} исследовать свойства постоянных неодимовых магнитов;
измерить с их помощью горизонтальную и вертикальную составляющие индукции
магнитного поля Земли и магнитное наклонение.

\textbf{В работе используются:} неодимовые магниты; тонкая нить для
изготовления крутильного маятника; медная проволока; электронные весы;
секундомер; измеритель магнитной индукции; штангенциркуль; брусок, линейка
и штатив из немагнитных материалов; набор гирь и разновесов.

\section*{Теория}

\subsection*{Магнитный диполь}
Магнитный момент магнитного диполя может быть рассчитан по формуле:
\[\mathbf{m} = I\mathbf{S}\]
Поле точечного диполя:
\[\textbf{B}_\text{дип} = \frac{\mu_0}{4\pi}\left(\frac{3(\mathbf{m} \cdot \textbf{r})\textbf{r}}{r^5} - \frac{\textbf{m}}{r^3}\right)\]
Во внешнем магнитном поле с индукцией $\textbf{B}$ на точеный магнитный
диполь $\mathbf{m}$ действует механический момент сил
$$\textbf{M}=[\mathbf{m}, \textbf{B}]$$
При этом потенциальная энергия которой обладает диполь с постоянным $\mathbf{m}$, равна
$$W = -(\mathbf{m} \cdot \textbf{B})$$
Когда диполь ориентирован вдоль внешнего поля, он находится в состоянии
равновесия. При этом если $\mathbf{m} \uparrow \uparrow B$, то равновесие
устойчивое (минимум энергии), если $\mathbf{m} \uparrow \downarrow B$, то
равновесие неустойчивое (максимум энергии).
В неоднородном поле на магнитный диполь действует сила:

\[\textbf{F} = (\mathbf{m} \cdot \nabla)\textbf{B},\]
В частности, проекция на ось $x$ имеет вид
\begin{equation*}
	F_x = \mathbf{m}_x\frac{\partial B_x}{\partial x} + \mathbf{m}_y\frac{\partial B_x}{\partial y} + \mathbf{m}_z\frac{\partial B_x}{\partial z}.
\end{equation*}
То есть магнитный диполь в неоднородном поле ориентируется вдоль силовых линий и втягивается в область сильного поля.\\
Сила, с которой взаимодействуют 2 магнита, оси которых сонаправлены:
\begin{equation}
	\label{6}
	F_{12}= \mathbf{m}_1 \frac{\partial{B_2}}{\partial{r}} = \mathbf{m}_1\frac{\partial{(2\mathbf{m}_2/r^3)}}{\partial{r}} = -\frac{6 \mathbf{m}_1\mathbf{m}_2}{r^4} \;(\text{ед}.\; \text{СГС}).
\end{equation}
(при использовании системы СИ нужно домножить на $\mu_0 / 4\pi$). Здесь
магниты притягиваются, если их магнитные моменты сонаправлены, и
отталкиваются, если направлены противоположно.


\subsection*{Неодимовые магниты}

В работе используются неодимовые магниты шарообразной формы. Магнитное поле
намагниченного шара на расстояниях $r \geq R$ от центра шара совпадает с
полем точечного магнитного диполя, расположенного в центре, магнитный
момент $\mathbf{m}$ которого совпадает с полным моментом шара. Внутри шара
магнитное поле однородно и равно
\[\textbf{B}_{0} = \frac{\mu_0 \mathbf{m}}{2\pi R^3}\]

В качестве ещё одной характеристики материала магнита используют остаточную
намагниченность $\textbf{M}$. $$\mathfrak{m}= \textbf{M}V ,$$ где
$\displaystyle V = \frac{4\pi}{3}R^3$ - объём магнита. Величину
$\textbf{B}_r = \mu_0 \textbf{M}$ называют остаточной индукцией материала
(в СГСЭ $B_r = 4\pi \mathbf{M}$). Из сказанного выше нетрудно видеть, что
индукция $\textbf{B}_p$ \textit{на полсюсах} однородно намагниченного шара
направлена по нормали к поверхности и совпадает поэтому с индукцией внутри
шара $\textbf{B}_p = \textbf{B}_0$. Величина $B_p$ связана с остаточной
индукцией $B_r$ соотношением \[B_p = B_o = \frac{2}{3}B_r \]


\subsection*{Определение магнитного момента магнитных шариков}

Величину магнитного момента $\mathfrak{m}$ двух одинаковых шариков можно
рассчитать, зная их массу $m$ и определив максимальное расстояние
$r_{max}$, на котором они еще удерживают друг друга в поле тяжести. При
максимальном расстоянии сила тяжести шариков $mg$ равна силе их магнитного
притяжения.

% \begin{wrapfigure}{r}{0.25\textwidth}
% 		\includegraphics[width=0.4\textwidth, height = 0.2\textheight]{rmax (1)}
% \end{wrapfigure}

Когда векторы двух магнитных моментов ориентированы вертикально, имеем

\begin{equation*}
	\mathbf{m} = \sqrt{\frac{2\pi mgr^4_{max}}{3 \mu_0}} \ \text{(ед. СИ).}
\end{equation*}

\section*{Результаты измерений и обработка данных}
В начале были проведены подготовительные измерения, данные которых приведены в таблице.
\begin{table}[H]
	\centering
	\begin{tabular}{|c|c|c|}
		\hline
		$m,\ \mili\gram$ & $B_{p}$, мТл & $d,\ \mm$   \\
		\hline
		$830 \pm 0.001$  & $600 \pm 1$  & $6 \pm 0.1$ \\
		\hline
	\end{tabular}
	\caption{Подготовительные измерения}
\end{table}

\subsection*{Измерение магнитного момента шариков}
\subsection*{Метод А}
Предельное расстояние, на котором шарики удерживали друг друга в поле силы тяжести:
\[
	r_{max} = (2.2 \pm 0.1) \cm
\]
Тогда по формуле
$\mathbf{m} = \sqrt{\frac{r_{max}^4 m g}{6}}$
\[
	\mathbf{m} = (57 \pm 2)\ \text{ед. СГС} \ (\varepsilon = 3 \%)
\]

\subsection*{Метод Б}
Найдем массу гири, при коротой цепочка отрывается:
\[
	m \approx (220 \pm 5) \gram
\]
$ F = m_{max} g \approx 1.08\ F_0 = \frac{3 \mathbf{m}^2}{8 R^4} $ \\
$\mathbf{m} = \sqrt{\frac{d^4 m_{max} g}{6.5}}$
\[
	\mathbf{m} = (66 \pm 2)\ \text{ед. СГС} \ (\varepsilon = 3 \%)
\]
Рассчитаем намагниченность материала:
\[
	\mathbf{M} = \frac{\mathbf{m}}{V} = (560 \pm 15)\ \text{ед. СГС}
\]
Тогда остаточная индукция:
\[
	B_r = 4\pi M = (7000 \pm 200)\ \text{ед. СГС}
\]
Можем также рассчитать индукцию на полюсе магнита:
\[
	B_p = \frac{2}{3}B_r = (4600 \pm 100)\ \text{ед. СГС}
\]

\subsection*{Определение горизонтальной проекции магнитного поля Земли}

Была собрана установка для измерения периода малых колебаний магнитной стрелки.
Данные измерений приведены в таблице. По данным из таблицы строим график
T(n).\\

\begin{table}[h]
	\centering
	\begin{tabular}{|l|c|c|c|}
		\hline  $n_{\text{шар}}$ & $5T$, с & $T$, с \\ \hline
		3                        & 4,0     & 0,8    \\ \hline
		4                        & 5,8     & 1,1    \\ \hline
		5                        & 6,5     & 1,3    \\ \hline
		6                        & 8,5     & 1,7    \\ \hline
		7                        & 10,3    & 2,1    \\ \hline
		8                        & 12,0    & 2,4    \\ \hline
		9                        & 13,7    & 2,8    \\ \hline
		10                       & 14,9    & 3,0    \\ \hline
		11                       & 16,5    & 3,2    \\ \hline
		12                       & 18,0    & 3,6    \\ \hline
	\end{tabular}
	\caption{Зависимость периода колебаний от количества шариков в магнитной стрелке}
\end{table}

\begin{figure}[H]
	\centering
	\includesvg[width=0.9\linewidth]{figures/T(n).svg}
\end{figure}

При малых колебаниях :
\[J_n\theta'' + \mathbf{m}_nB_h \theta = 0, \ \ J_n = \frac{1}{3}n^3mR^2, \ \ \mathbf{m}_n = \mathbf{m} \cdot n  \]
Отсюда находим период колебаний:

\begin{equation*}
	T(n) = 2\pi \sqrt{\frac{mR^2}{3 \mathfrak{m} B_{h}}} \cdot n, \quad \quad \frac{T(n)}{n} = k,
	\quad \quad k =0.31
\end{equation*}
Отсюда находим горизонтальную состовляющую магнитного поля земли:
\[B_h = \frac{4\pi^2 mR^2}{3k^2\mathbf{m}} = (1,81 \pm 0,49) \cdot 10^{-5} \ \ \text{Тл} \ \ (\varepsilon = 25\%)\]

\newpage
\subsection*{Определение вертикальной проекции магнитного поля Земли}
Подвешиваем четное число шариков за центр и пытаемся уравновесить момент сил магнитного поля с помощью дополнительных грузиков в виде проволок. Были получены следующие данные. По полученным данным был построен график.

\begin{table}[H]
	\centering
	\begin{tabular}{|c|c|c|c|}
		\hline $n_{\text{шар}}$ & $m_{\text{гр}}$, г & $r_{\text{гр}}$, см & $\mathcal{M}$, H $\cdot $ м $\cdot 10^{-5}$ \\ \hline
		4                       & 0,35               & 0,6                 & 2,1                                         \\ \hline
		6                       & 0,19               & 1,2                 & 2,7                                         \\ \hline
		8                       & 0,17               & 1,8                 & 3,1                                         \\ \hline
		10                      & 0,16               & 2,4                 & 3,8                                         \\ \hline
		12                      & 0,14               & 3                   & 4,3                                         \\ \hline
	\end{tabular}
	\caption{Зависимость момента от количества шариков в <<магнитной стрелке>>}
\end{table}

\begin{figure}[H]
	\centering
	\includesvg[width=0.9\linewidth]{figures/M(n).svg}
\end{figure}

Здесь принята погрешность измерения уравновешивающей массы за $5\%$. Выразим вертикальную составляющую магнитного поля:
\[\mathcal{M}_n = m_{\text{гр}}gr_{\text{гр}} = n\mathbf{m}B_v\]
\[B_v = \frac{\mathcal{M}_n}{n\mathbf{m}} = (5,54 \pm 0,69) \cdot 10^{-5} \ (\varepsilon = 9,6\%)\]

Теперь можем посчитать полную величину индукции и магнитное наклонение:

\[|B| = \sqrt{B_h^2 + B_v^2} = (5,83 \pm 0,66) \cdot 10^{-5} \ \ \text{Тл} \ (\varepsilon = 11\%) \]

\[\tg{\beta} = \frac{B_v}{B_h} = 3,07 \pm 0,49 \ \ (\varepsilon = 16\%)\ \ \rightarrow \beta = 72,0^{\circ} \pm 8,9^{\circ} \ \ (\varepsilon = 13,5\%)\]

Можно также рассчитать теоретическое значение $\beta$ на широте москвы $\varphi = 56^{\circ}$ в предположении, что Земля - однородно намагниченный шар:

\[\beta = \arcctg{\frac{\frac{2P_m \sin{\varphi}}{R^3}}{\frac{-P_m \cos{\varphi}}{R^3}}} \approx 71^{\circ}\]

Табличные данные магнитного поля в Москве приведены ниже:
\[B_{\text{табл}} = 5 \cdot 10^{-5} \ \text{Тл} \ \ \ \beta_{\text{табл}} \approx 74^{\circ} - 78^{\circ}\]

\section*{Выводы}
В данной лабораторной работе были изучены характеристики неодимовых магнитов, а также вертикальная и горизонтальная составляющая магнитного поля. \\
\begin{enumerate}
	\item Остаточная намагниченность $B_p = (4600 \pm 100)\ \text{ед. СГС}$ совпадает по
	      порядку с измеренными значениями $B_{p_{\text{табл}}} \approx 6000\ \text{ед. СГС}$.
	\item Значения $B_h$ и $B_v$ также совпадают по порядку с табличными. По
	      значению могут не совпадать, так как в кабинете еще есть электронные
	      устройства, имеющие также своё магнитное поле. Полная величина индукции
	      совпадает с табличными данными в пределах погрешности.
	\item Погрешности в измерениях компонент магнитного поля высоки ввиду
	      неточности экспериментa.
	\item Магнитное наклонение совпадает с табличными данными в пределах
	      погрешности.
\end{enumerate}
\end{document}

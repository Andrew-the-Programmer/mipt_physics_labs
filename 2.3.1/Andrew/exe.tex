\documentclass[a4paper,12pt]{article}

\usepackage{other/lab_preamble}

\graphicspath{{data/}}

\begin{document}

\section{Ход работы}

\subsection{Определение объемов форвакуумной и высоковакуумной частей установки}

\begin{enumerate}

  \item Перед началом работы проверим, что все краны приведены в правильное положение.

  \item Запустим воздух в систему (для этого нужно открыть кран $К_2$  и подождать пару минут пока воздух заполнит установку).

  \item Запустим форвакуумный насос, чтобы он откачал воздух из установки.

        Пронаблюдаем за тем, как давление в установке уменьшается и продолжим откачку до момента, пока давление не будет порядка $ 10^{-2}~торр$.

  \item Отсоединим установку от форвакуумного насоса, а затем объем, заключенный в кранах и капиллярах форвакуумной части, откроем на всю форвакуумную часть. Тогда давление изменится

  \item Запишем показания масляного манометра, а именно высоту масла в обоих коленах:
        \begin{align}
          h_1 = (34.9 \pm 0.1) ~см, &  & h_2 = (6.2 \pm 0.1) ~см,
        \end{align}
        \begin{align}
          \sigma_{\Delta h} = \sqrt{\sigma_{h1}^2 + \sigma_{h2}^2}\approx 1.6~\% &  &
          \Delta h_{фв} = (28.7 \pm 0.5) ~см.
        \end{align}

  \item Зная объем "запертой"  части установки $V_{кап} = 50 ~см^3$ и используя соотношение $P_\text{А} V_{кап}=P_2 V_2$ вычислим объем форвакуумной части установки. При этом давление $P_1 = P_{атм} = (98.4 \pm 0.1) ~кПа$ $P_2 =   \rho_{масл} g \Delta h_{фв}$, а относительная погрешность полученного значения равна относительной погрешности величины $\Delta h_{фв} $:
        $$\varepsilon_V = \varepsilon_{P_1} \approx 1.6 ~\%.$$ и в результате имеем:

        \begin{equation}
          V_{фв} = (1.97 \pm 0.03)~ л
        \end{equation}


  \item Проведем те же самые измерения с диффузионным насосом и получим объем установки, из которой вычитанием объема форвакуумной части получается объем высоковакуумной части.
        \begin{align}
          h_3 = (29.8 \pm 0.1) ~см, &  & h_4 = (11.6 \pm 0.1) ~см,
        \end{align}
        \begin{align}
          \Delta h_{полн} = (18.2 \pm 0.2) ~см.
        \end{align}

        Погрешности высот определяются аналогично предыдущему пункту. Как и формула для полного объема установки, тогда:
        \begin{align}
          V_\text{полн} = \frac{P_\text{А}}{\rho g \Delta h_\text{полн}} V_{кап} \approx 3.11~л, &  &
          \varepsilon_{V_{полн}} = \varepsilon_{\Delta h} \approx 1~\%.
        \end{align}

        В результате искомая величина равна:
        \begin{align}
          V_{вв} = V_{полн} - V_{фв} = 1.14~л, &  & \sigma_{V_{вв}} = \sqrt{\sigma_{V_{полн}}^2+ \sigma_{V_{фв}}^2} \approx 0.04~л,
        \end{align}
        \begin{align}
          V_{вв} = (1.14 \pm 0.04)~л.
        \end{align}

\end{enumerate}

\subsection{Получение высокого вакуума и измерение скорости откачки}

\begin{enumerate}
  \setcounter{enumi}{7}

  \item Не выключая форвакуумного насоса убедимся в том, что в установке не осталось запертых объемов.

  \item Откачав установку до давления порядка $ 10^{-2}~торр$, приступим к откачке ВБ с помощью диффузионного насоса.


  \item С помощью термопарного манометра пронаблюдаем за тем, как идет откачка ВБ. Мы должны продолжать процесс откачки до тех пор, пока там не установится давление порядка $3 \cdot 10^{-4}~торр.$ При приближении давления к этой величине масло в диффузионном насосе закипит, поэтому подсчитаем количество капель, стекающих из сопла второй ступени диффузионного насоса: $$ N = 10 ~ капель.$$

  \item С помощью ионизационного манометра измерим значение предельного давления в системе со стороны высоковакуумной части: $$P_{пр} = (8.3 \pm 0.1)  \cdot 10^{-5} ~торр.$$


  \item Найдем скорость откачки по ухудшению и улучшению вакуума, для этого открывая и закрывая кран $К_3$ будем то подключать насос к объему, то отключать его, при этом на видео зафиксируем показания манометра от времени и построим графики необходимых  зависимостей (каких именно подробнее описано в соответствующих пунктах ниже), для которых определим коэффициенты наклона прямых и их погрешности (с помощью МНК).

        Для случая улучшения вакуума воспользуемся формулой \eqref{exp} и построим график зависимости $(ln(P-P_{пр}))$ от $t$. При построении такого графика из МНК получим коэффициент наклона --- $k$, с помощью которого можно найти $W = -kV_{вв}$. Построим эти графики (Рис. \ref{graph2}, \ref{graph3}, \ref{graph4}):
  \item Оценим величину потока газа  $Q_Н$. Для этого воспользуемся данными, полученными при ухудшении вакуума. А именно построим графики зависимости $P(t)$ и определим для них коэффициенты угла наклона прямой. Поскольку $V_{вв}dP = (Q_Д + Q_И) dt$ получим $(Q_Д + Q_И) = kV_{вв}$. По графикам получаем:
        Используя формулу $Q_Н = P_{пр}W - (Q_Д + Q_И)$, а значит $\varepsilon_{Q_Н} =  \sqrt{\varepsilon_{P_{пр}W}^2 + \varepsilon^2} \approx 10.4\%$ получим, что: $Q_Н = (5.4 \pm 0.08) \cdot 10^{-6} ~ торр \cdot л / с.$



  \item Оценим пропускную способность трубки по формуле (6):
        \begin{align}
          L = (10 \pm 1)~ см; &  & d = (0.8 \pm 0.1) ~ мм,
        \end{align}

        \begin{equation}
          C_{тр} = (2.1 \pm 0.1)~л / с.
        \end{equation}

        Погрешность $C_{тр}$  оценена как корень из суммы квадратов погрешностей длинны и диаметра (которые явным образом не указаны на установке, оценка довольно грубая).

        Скорость откачки по порядку сходится с пропускной способностью трубки, что означает -- эксперимент достаточно успешен.

  \item Введем в систему искусственную течь и запишем значение  установившегося при этом давления и давления $P_{фв}$:

        \begin{align}
          P_{уст} = (1.2 \pm 0.1) \cdot 10^{-4} ~ торр. &  & P_{фв} = (1.6 \pm 0.1) \cdot 10^{-3} ~ торр.
        \end{align}


  \item Поскольку
        $$P_{пр} W = Q_1, \quad P_{уст} W = Q_1 + \frac{d(PV)_{кап}}{dt},$$
        то с учетом \eqref{ty}, получаем:
        \begin{equation}
          W = \frac{P_{фв}}{P_{уст}-P_{пр}}\frac{4r^3}{3L}\sqrt{\frac{2\pi RT}{\mu}} \approx 0.046~\frac{л}{с}
        \end{equation}

        (Поскольку давления померены с точностью не менее $10\%$, то можно учитывать погрешность, вносимую величиной $\frac{d(PV)_{кап}}{dt}$ относительная погрешность которой равна относительной погрешности $C_{тр}$, то есть составляет $5\%$)


  \item Следуя указаниям в методичке выключаем установку.

\end{enumerate}

\end{document}
